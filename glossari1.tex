%\newglossaryentry{fattoreng}{name={Fattore},description={unità di misura 
%dell'angolo che non fa parte del SI. Il gon è pari a 1/400 parte dell'angolo 
%giro.}}
\newglossaryentry{gong}{name={Gon},description={}}
\newacronym{ASCII}{ASCII}{American Standard Code for Information Interchange}
\newacronym{DICE}{DICE}{Discordi Interni Concordi Esterni}
\newacronym{Lim}{Lim}{Limite}
\newacronym{Log}{Log}{Logaritmo in base $10$}
\newacronym{MCD}{MCD}{Massimo comune divisore}
\newacronym{Re(z)}{Re(z)}{Parte reale}
\newacronym{acos}{acos}{arco coseno}
\newacronym{asin}{asin}{arco seno}
\newacronym{atan}{atan}{arco tangente}
\newacronym{cos}{cos}{coseno}
\newacronym{daga}{dag}{decagrammo}
\newacronym{dala}{dal}{decalitro}
\newacronym{dama}{dam}{decametro}
\newacronym{dla}{dl}{decilitro}
\newacronym{hga}{hg}{ettogrammo}
\newacronym{hla}{hg}{ettolitro}
\newacronym{hma}{hg}{ettometro}
\newacronym{kBa}{kB}{kilo byte}
\newacronym{kga}{kg}{kilogrammo}
\newacronym{kma}{hg}{kilometro}
\newacronym{lb}{lb}{Logaritmo in base $2$}
\newacronym{ln}{ln}{Logaritmo in base $e$}
\newacronym{mcm}{mcm}{Minimo comune multiplo}
\newacronym{rad}{rad}{radiante}
\newacronym{sin}{sin}{seno}
\newacronym{tan}{tan}{tangente}
\newacronym{SI}{SI}{Sistema internazionale di unità di misura}
\newglossaryentry{Ag}{type=symbols,name={A},description={Il numero dieci in 
base 16}}
\newglossaryentry{Campoesistenzag}{name={Campo di 
esistenza},description={Sinonimo per insieme di 
definizione},see={dominiog,InsiemeDefinzioneg}}
\newglossaryentry{GBg}{type=symbols,name={GB},description={gigabyte}}
\newglossaryentry{Gg}{type=symbols,name={G},description={giga}}
\newglossaryentry{Gig}{type=symbols,name={Gi},description={gigabinary}}
\newglossaryentry{InsiemeDefinzioneg}{name={Insieme di 
definizione},description={per una funzione, è il sottoinsieme del dominio in 
cui la funzione è effettivamente definita},see={dominiog}}
\newglossaryentry{Kig}{type=symbols,name={Ki},description={kilobinary}}
\newglossaryentry{Mgg}{type=symbols,name={Mg},description={megagrammo}}
\newglossaryentry{Mg}{type=symbols,name={M},description={mega}}
\newglossaryentry{Mig}{type=symbols,name={Mi},description={megabinary}}
\newglossaryentry{Petag}{name={Peta},description={prefisso moltiplicativo che 
indica $10^{15}$}}
\newglossaryentry{Pg}{type=symbols,name={P},description={Peta}}
\newglossaryentry{Progressionearitm}{name={Progressione 
aritmetica},description={Progressione numerica per cui è costante la differenza 
tra un ternime e il suo predecessore}}
\newglossaryentry{Progressionegeom}{name={Progressione 
},description={Progressione numerica per cui è costante il rapporto tra un 
ternime e il suo predecessore}}
\newglossaryentry{Razionalizzazioneg}{name={Razionalizzazione},description={procedura
 che cerca di trasformazre una frazione che ha un radicale al denominatore in 
una frazione che ha un radicale solo al numeratore}}
\newglossaryentry{TBg}{type=symbols,name={TB},description={terabyte}}
\newglossaryentry{Tgg}{name={tera},description={prefisso moltiplicativo che 
indica $10^{12}$}}
\newglossaryentry{Tg}{type=symbols,name={T},description={tera}}
\newglossaryentry{Tig}{type=symbols,name={Ti},description={terabinary}}
\newglossaryentry{Unioneindisiemig}{name={Unione di insiemi},description={dati 
due o più insiemi, è l'insieme formato dagli elementi che appartengono a tutti 
gli insiemi},see={insiemeg}}
\newglossaryentry{acosg}{name={acos},description={abbreviazione di arco coseno, 
funzione inversa del coseno}}
\newglossaryentry{addendog}{name={Addendo},description={termine dell'addizione}}
\newglossaryentry{addizioneg}{name={Addizione},description={operazione binaria}}
\newglossaryentry{adiacenteg}{name={Adiacente},description={lato 
triangolo},see={latoadiacenteg,angoloadiacenteg,segmentoadiacenteg}}
\newglossaryentry{ag}{type=symbols,name={a},description={atto}}
\newglossaryentry{albbing}{name={Albero binario},description={un albero binario 
è formato da un nodo detto radice da cui si staccano due nodi detti figli. Un 
nodo senza figli è detto foglia}}
\newglossaryentry{altezzag}{name={Altezza},description={in un triangolo, 
segmento per un vertice che interseca perpendicolarmente la retta in cui giace 
il lato opposto al vertice},see={segmentog,ortocentrog}}
\newglossaryentry{andg}{name={And},description={operatore dell'algebra di Boole 
che è vero solo quando gli ingressi sono entrambi veri}}
\newglossaryentry{andsg}{type=symbols,name={\ensuremath{\wedge}},description={and}}
\newglossaryentry{ands}{name={and},description={operatore logico. Vale uno solo 
se in entrata sono entrambi a uno altrimenti vale zero}}
\newglossaryentry{angolnegativoog}{name={Angolo negativo},description={un 
angolo che per costruirlo si è utilizzato un movimento orario}}
\newglossaryentry{angoloacutog}{name={Angolo acuto},description={un angolo che 
è minore di un angolo retto}}
\newglossaryentry{angoloadiacenteg}{name={Angolo adiacente},description={due 
angoli sono adiacenti se hanno un lato appartenente alla stessa retta e stesso 
vertice}}
\newglossaryentry{angoloampiezzag}{name={Angolo ampiezza},description={misura 
dell'angolo. Ad ogni angolo è possibile associare un numero che varia con 
l'unità di misura}}
\newglossaryentry{angolocomplementareg}{name={Angolo 
complementare},description={due angoli sono complementari se la loro somma è un 
angolo retto},see={angolorettog}}
\newglossaryentry{angoloconsecutivog}{name={Angolo 
consecutivo},description={due angoli sono consecutivi se hanno un lato in 
comune e stesso vertice}}
\newglossaryentry{angoloesplementareg}{name={Angolo 
esplementare},description={due angoli sono esplementari se la loro somma è un 
angolo giro.},see={angolopiattg}}
\newglossaryentry{angologirog}{name={Angolo giro},description={un angolo in cui 
i due lati coincidono ha solo punti interni}}
\newglossaryentry{angolog}{name={Angolo},description={parte di piano compresa 
fra due semirette che hanno la stessa origine}}
\newglossaryentry{angolonullog}{name={Angolo nullo},description={un angolo in 
cui i due lati coincidono non ha punti interni}}
\newglossaryentry{angolopiattg}{name={Angolo piatto},description={un angolo che 
è la metà di un angolo giro}}
\newglossaryentry{angolopositivog}{name={Angolo positivo},description={un 
angolo che per costruirlo si è utilizzato un movimento antiorario}}
\newglossaryentry{angolorettog}{name={Angolo retto},description={un angolo che 
è la metà di un angolo piatto}}
\newglossaryentry{angolosupplentareg}{name={Angolo 
supplemetare},description={due angoli sono supplementari se la loro somma è un 
angolo piatto.},see={angologirog}}
\newglossaryentry{angolottusog}{name={Angolo ottuso},description={un angolo che 
è maggiore angolo retto}}
\newglossaryentry{antecedenteg}{name={Antecedente},description={primo e terzo 
termine di una proporzione}}
\newglossaryentry{antiorariog}{name={Antiorario},description={movimento 
circolare che avviene in modo opposto a quello delle lancette dell'orologio}}
\newglossaryentry{arcog}{name={Arco},description={parte di circonferenza o di 
curva delimitato da due punti detti estremi}}
\newglossaryentry{asing}{name={asin},description={abbreviazione di arco seno, 
funzione inversa del seno}}
\newglossaryentry{asintoticog}{name={Asintotico},description={quantità che si 
avvicina con un'approssimazione comunque piccola a un valore dato}}
\newglossaryentry{asintotog}{name={Asintoto},description={retta che si avvicina 
indefinitamente ad una curva data}}
\newglossaryentry{assesegg}{name={Asse segmento},description={retta che passa 
per il punto medio di un segmento, perpendicolare al segmento 
dato},see={puntomediog}}
\newglossaryentry{atang}{name={atan},description={abbreviazione di arco 
tangente, funzione inversa della tangente}}
\newglossaryentry{attog}{name={atto},description={prefisso moltiplicativo che 
indica $10^{-18}$}}
\newglossaryentry{baricentrog}{name={Baricentro},description={in un triangolo, 
punto di intersezione fra le mediane},see={medianag}}
\newglossaryentry{bg}{type=symbols,name={B},description={Il numero undici in 
base 16}}
\newglossaryentry{binomiog}{name={Binomio},description={espressione algebrica 
ottenuta dalla somma o dalla differenza di due monomi non 
simili},see={monomiosimileg}}
\newglossaryentry{bisettriceg}{name={Bisettrice},description={retta che passa 
per il vertice di un angolo e lo divide in due parti uguali}}
\newglossaryentry{bitg}{type=symbols,name={b},description={bit}}
\newglossaryentry{byteg}{type=symbols,name={B},description={byte}}
\newglossaryentry{catetog}{name={Cateto},description={in un triangolo 
rettangolo sono i due segmenti che formano l'angolo retto}}
\newglossaryentry{centig}{name={centi},description={prefisso moltiplicativo che 
indica $10^{-2}$}}
\newglossaryentry{centrocircoferenzag}{name={Centro 
circonferenza},description={punto equidistante da tutti punti della 
circonferenza}}
\newglossaryentry{cerchiog}{name={Cerchio},description={figura piana delimitata 
da una circonferenza}}
\newglossaryentry{cfg}{type=symbols,name={c},description={centi}}
\newglossaryentry{cg}{type=symbols,name={C},description={Il numero dodici in 
base 16}}
\newglossaryentry{circgoniog}{name={Circonferenza 
goniometrica},description={circonferenza con centro nell'origine e di raggio 
unitario}}
\newglossaryentry{circocentrog}{name={Circocentro},description={in un triangolo 
punto di incontro tra i tre assi dei lati},see={assesegg}}
\newglossaryentry{circonferenzag}{name={Circonferenza},description={inseme dei 
punti equidistanti da un punto detto centro}}
\newglossaryentry{cnug}{type=symbols,name={C},description={insieme dei numeri 
complessi},see={numcompg}}
\newglossaryentry{coefficienteangolareg}{name={Coefficiente 
angolare},description={nella geometria analitica rappresenta il coefficiente 
legato all'inclinazione di una retta rispetto all'asse delle ascisse}}
\newglossaryentry{complmentareindisiemig}{name={Insieme 
complementare},description={dato un insieme, il suo complementare è l'insieme 
di tutti gli elementi che non gli appartengono},see={insiemeg}}
\newglossaryentry{comunqueg}{type=symbols,name={\ensuremath{\forall}},description={comunque}}
\newglossaryentry{concag}{name={Concavo},description={},see={figconcag}}
\newglossaryentry{concordenumerog}{name={Concordi},description={sono due numeri 
che hanno lo stesso segno}}
\newglossaryentry{congruenzag}{name={Congruenza},description={in geometria 
sinonimo di uguaglianza}}
\newglossaryentry{coniunumcompg}{name={Numero coniugato},description={il 
coniugato di un numero complesso è il numero con la parte immaginaria opposta 
${z=a-\uimm b}$},see={unitaimmaginariag}}
\newglossaryentry{conseguenteg}{name={Conseguente},description={secondo e 
quarto termine di una proporzione}}
\newglossaryentry{convesg}{name={Convesso},description={},see={figconveg}}
\newglossaryentry{cooordinatecart}{name={Coordinate 
Cartesiane},description={nel piano sistema formato da due rette incidenti in un 
punto detto origine. Le due rette sono dette assi cartesiani, su ciascun asse è 
definito un sistema di ascisse assegnando su ciascun asse un'unità e l'origine 
nel punto di intersezione},see={sistemascisseg}}
\newglossaryentry{coorpolog}{name={Coordinate polari},description={un sistema 
di coordinate polari individua la posizione di un punto $P$ nel piano, tramite 
una coppia $(\rho;\theta)$ dove il primo numero è la distanza del punto $P$ 
detto polo e da un angolo $\theta$ misurato in senso antiorario da una 
semiretta di origine $P$},see={distanzag}}
\newglossaryentry{cordag}{name={Corda},description={segmento che congiunge due 
punti di una circonferenza},see={circonferenzag,segmentog}}
\newglossaryentry{cosenog}{name={Coseno},description={in un triangolo 
rettangolo il coseno di un anglo acuto è uguale al rapporto tra il cateto 
adiacente all'angolo e l'ipotenusa. In una circonfereza goniometrica, il coseno 
di un angolo è l'ascissa del punto individuato dall'angolo sulla 
circonferenza},see={latoadiacenteg,circgoniog,ipotenusag,catetog}}
\newglossaryentry{costanteg}{name={Costante},description={una costante è un 
carattere che rappresenta una quantità numerica non nota ma fissata}}
\newglossaryentry{crg}{type=symbols,name={C},description={Il numero cento nel 
sistema di numerazione romano}}
\newglossaryentry{dag}{type=symbols,name={da},description={deca}}
\newglossaryentry{decag}{name={deca},description={prefisso moltiplicativo che 
indica $10$}}
\newglossaryentry{decig}{name={deci},description={prefisso moltiplicativo che 
indica $10^{-1}$}}
\newglossaryentry{deltag}{name={Delta},description={ha molti significati in 
matematica},see={discriminanteg}}
\newglossaryentry{deltasg}{type=symbols,name={\ensuremath{\Delta}},description={discriminante
 equazione di secondo grado $\Delta=b^2-4ac$}}
\newglossaryentry{denominatoreg}{name={Denominatore},description={termine 
divisore in una frazione}}
\newglossaryentry{derivatag}{name={Derivata},description={limite, se esiste ed 
è finito,  per $h\to 0$ del rapporto incrementale},see={rapportoincrementaleg}}
\newglossaryentry{dfg}{type=symbols,name={d},description={deci}}
\newglossaryentry{dg}{type=symbols,name={D},description={numero tredici in base 
16}}
\newglossaryentry{diagonaleg}{name={Diagonale},description={in un poligono 
regolare, segmento che unisce due vertici non adiacenti}}
\newglossaryentry{diametrog}{name={Diametro},description={corda che passa per 
il centro di una figura chiusa},see={cordag}}
\newglossaryentry{differenzag}{name={Differenza},description={risultato 
sottrazione}}
\newglossaryentry{differenzaindisiemig}{name={Differenza di 
insiemi},description={dati due  insiemi, è l'insieme formato dagli elementi che 
appartengono ad uno ma non all'altro},see={insiemeg}}  
\newglossaryentry{discordinumerog}{name={Discordi},description={sono due numeri 
che hanno segno diverso}}
\newglossaryentry{discriminanteg}{name={Discriminante},description={in un 
polinomio di secondo grado $ax^2+bx+c$ il discriminante è $\Delta=b^2-4ac$}}
\newglossaryentry{disparig}{name={Dispari},description={numero non divisibile 
per due}}
\newglossaryentry{distanzag}{name={Distanza},description={esprime la misura 
della lontananza di due punti. La distanza tra due punti nel piano, si ottiene 
utlizzando il teorema di Pitagora. In forma algebrica 
$d(AB)=\sqrt{(x_1-x_0)^2+(y_1-y_0)^2}$}}
\newglossaryentry{distintog}{name={Distinto},description={non identico}}
\newglossaryentry{disuguaglianzag}{name={Disuguaglianza},description={una delle 
seguenti relazioni: $a<b$ a minore di b, $a>b$ a maggiore di b, $a\leq b$ a 
minore o uguale a b, $a\geq b$ a maggiore o uguale a b}}
\newglossaryentry{dividendog}{name={Dividendo},description={primo termine della 
divisione}}
\newglossaryentry{divisioneg}{name={Divisione},description={operazione binaria 
inversa della moltiplicazione}}
\newglossaryentry{divisoreg}{name={Divisore},description={secondo termine della 
divisione}}
\newglossaryentry{dominiog}{name={Dominio},description={per una funzione, è 
l'insieme dei valori per cui essa è  definita}}
\newglossaryentry{drg}{type=symbols,name={D},description={numero 500 nel 
sistema di numerazione romano}}
\newglossaryentry{eg}{type=symbols,name={E},description={Il numero quattordici 
in base 16}}
\newglossaryentry{elevamentoapotenzag}{name={Elevamento a 
potenza},description={operazione che associa ad un numero detto base ad un 
numero naturale detto esponente, il risultato detto potenza è uguale al 
prodotto della base per tante volte quanto è l'esponente},see={potenzag}}
\newglossaryentry{equazioneDetg}{name={Equazione determinata},description={è 
un'equazione con un numero finito di soluzioni}}
\newglossaryentry{equazioneIdentg}{name={Equazione 
indeterminata},description={è un'equazione con un numero infinito di soluzioni}}
\newglossaryentry{equazioneImpg}{name={Equazione impossibile},description={è 
un'equazione che non ha soluzioni}}
\newglossaryentry{equazionealgebricag}{name={Equazione 
algebrica},description={equazione scritta in forma polinomiale}}
\newglossaryentry{equazioneequig}{name={Equazioni 
equivalenti},description={sono equazioni che hanno le stesse soluzioni}}
\newglossaryentry{equazionefrag}{name={Equazione frazionaria},description={è 
un'equazione che ha al denominatore l'incognita}}
\newglossaryentry{equazioneg}{name={Equazione},description={uguaglianza 
condizionata fra due espressioni algebriche}}
\newglossaryentry{equazioneidentitag}{name={Equazione 
identità},description={equazione verificata da qualunque valore dell'incognita}}
\newglossaryentry{equazioneinterag}{name={Equazione 
intera},description={nessuna delle incognite appare mai al denominatore delle 
frazioni che la compongono o ha esponente negativo}}
\newglossaryentry{equazioneirraziog}{name={Equazione 
irrazionale},description={almeno una delle incognite appare come argomento di 
radice}}
\newglossaryentry{equazionenormaleg}{name={Forma normale 
equazione},description={una equazione è in forma normale se tutti i termini 
sono a primo membro ordinati mentre a destra vi è solo lo zero}}
\newglossaryentry{equazionepiuincogniteg}{name={Equazione in più 
incognite},description={equazione con un numero di incognite superiori ad uno}}
\newglossaryentry{equazioneraziog}{name={Equazione 
razionale},description={nessuna delle incognite appare come argomento delle 
radici}}
\newglossaryentry{equiangolog}{name={Equiangolo},description={figura piana con 
tutti gli angoli uguali}}
\newglossaryentry{equilaterog}{name={Equilatero},description={figura piana con 
tutti i lati uguali}}
\newglossaryentry{esagesimaleg}{name={Esadecimale},description={numero espresso 
nella notazione in base sedici}}
\newglossaryentry{esisteg}{type=symbols,name={\ensuremath{\exists}},description={esiste}}
\newglossaryentry{esisteunicog}{type=symbols,name={\ensuremath{\exists!}},description={esiste
 ed è unico}}
\newglossaryentry{esponenteg}{name={Esponente},description={nell'elevazione a 
potenza indica quante volte la base deve essere moltiplicata per se 
stessa.},see={elevamentoapotenzag}}
\newglossaryentry{estremog}{name={Estremo},description={primo e quarto termine 
di una proporzione}}
\newglossaryentry{estremoinferioreg}{name={Estremo inferiore},description={il 
più grande dei minoranti},see={minoranteg}}
\newglossaryentry{estremosuperioreg}{name={Estremo superiore},description={il 
più piccolo dei maggioranti},see={maggioranteg}}
\newglossaryentry{ettog}{name={etto},description={prefisso moltiplicativo che 
indica $100$}}
\newglossaryentry{fascog}{name={Fascio},description={insieme di rette che 
passano per lo stesso punto o un insieme di rette parallele}}
\newglossaryentry{fasenumcompg}{name={Fase},description=angolo che nel piano 
complesso un vettore forma con l'asse reale}
\newglossaryentry{fattoreg}{name={Fattore},description={termine della 
moltiplicazione. Intero che divide esattamente un intero dato}}
\newglossaryentry{fattorialeg}{name={Fattoriale},description={il prodotto dei 
numeri interi positivi minori o uguali a tale numero}}
\newglossaryentry{feg}{type=symbols,name={f},description={femto}}
\newglossaryentry{femtog}{name={femto},description={prefisso moltiplicativo che 
indica $10^{-15}$}}
\newglossaryentry{fg}{type=symbols,name={F},description={Il numero quindici in 
base 16}}
\newglossaryentry{figconcag}{name={Figura concava},description={una figura è 
concava se presi due qualunque suoi punti il segmento che li congiunge non 
appartiene alla figura}}
\newglossaryentry{figconveg}{name={Figura convessa},description={una figura è 
convessa se presi due qualunque suoi punti il segmento che li congiunge 
appartiene alla figura}}
\newglossaryentry{figurag}{name={Figura},description={ogni insieme di punti o 
di linee o di superfici}}
\newglossaryentry{frazineappag}{name={Frazione apparente},description={nella 
frazione apparente il numeratore è multiplo del denominatore}}
\newglossaryentry{frazinpropg}{name={Frazione impropria},description={nella 
frazione impropria il numeratore è maggiore e non multiplo del 
denominatore},see={multiplog}}
\newglossaryentry{frazionegeng}{name={Frazione generatrice},description={la 
frazione corrispondente ad un numero decimale dato}}
\newglossaryentry{frazionesempg}{name={Semplificare una 
frazione},description={semplificare una frazione significa dividere il 
numeratore e il denominatore per il loro Massimo Comune Divisore ($\mcd$). Per 
la proprietà invariantiva la frazione ottenuta è equivalente a quella data}}
\newglossaryentry{frazioniequivg}{name={Frazioni equivalenti},description={due 
frazioni sono equivalenti quando rappresentano lo stesso quoziente.}}
\newglossaryentry{frazpropg}{name={Frazione propria},description={nella 
frazione propria il numeratore è minore del denominatore}}
\newglossaryentry{frequenzaassog}{name={Frequenza assoluta},description={numero 
di individui di una classe}}
\newglossaryentry{frequenzarelag}{name={Frequenza relativa},description={numero 
di individui di una classe diviso l'intera popolazione}}
\newglossaryentry{funzionediparig}{name={Funzione dispari},description={una 
funzione $\funzione{f}{\R}{\R}$ è dispari se per ogni $x$ $f(-x)=-f(x)$}}
\newglossaryentry{funzioneparig}{name={Funzione pari},description={una funzione 
$\funzione{f}{\R}{\R}$ è pari se per ogni $x$ $f(x)=f(-x)$}}
\newglossaryentry{funzionerazionalefratg}{name={Funzione razionale 
fratta},description={funzione che può essere espressa come un quoziente fra 
polinomi}}
\newglossaryentry{funzionerazionaleintg}{name={Funzione razionale 
intera},description={funzione che può essere espressa come un polinomio}}
\newglossaryentry{funzionezerog}{name={Funzione, zero di},description={valore 
per cui una funzione è nulla}}
\newglossaryentry{gg}{type=symbols,name={g},description={grammo}}
\newglossaryentry{gigabinaryg}{name={gigabinary},description={pari a $2^{30}$}}
\newglossaryentry{gigabyteg}{name={gigabyte},description={pari a $10^{9}$byte}}
\newglossaryentry{gigag}{name={giga},description={prefisso moltiplicativo che 
indica $10^{9}$}}
\newglossaryentry{giornog}{name={Giorno},description={unità di misura degli 
intervalli di tempo che non fa parte del SI, un giorno  vale\num{24} ore 
\num{1440} minuti o \num{86400} secondi},see={orag,minutog,secondog,SI}}
\newglossaryentry{gradodecg}{name={Grado sessadecimale},description={Unità di 
misura dell'ampiezza di un angolo, è la trecento-sessantesima parte di un 
angolo giro, non accettata dal SI, i suoi sottomultipli sono espressi in forma 
decimale},see={radianteg,SI}}
\newglossaryentry{gradog}{name={Grado sessagesimale},description={Unità di 
misura dell'ampiezza di un angolo, è la trecento-sessantesima parte di un 
angolo giro, non accettata dal SI, i suoi sottomultipli sono il minuto e il 
secondo},see={radianteg,minutog,secondog,SI}}
\newglossaryentry{gradopolinomiog}{name={Grado polinomio},description={è 
l'esponente o la gomma degli esponenti più grande di un polinomio}}
\newglossaryentry{gradositemag}{name={Grado sistema},description={il grado di 
un sistema, di più equazioni in tante incognite, è il prodotto dei gradi delle 
equazioni del sistema ridotte in forma normale.}}
\newglossaryentry{grandezzafisicag}{name={Grandezza fisica},description={è una 
qualunque proprietà della materia che può essere misurata (quantificata)}}
\newglossaryentry{hag}{type=symbols,name={G},description={etto}}
\newglossaryentry{hg}{type=symbols,name={h},description={ora}}
\newglossaryentry{immaginariopurog}{name={Immaginario puro},description={numero 
complesso con solo la parte immaginaria},see={numcompg}}
\newglossaryentry{incentrog}{name={Incentro},description={punto di incontro 
delle bisettrici di un triangolo},see={bisettriceg}}
\newglossaryentry{incognitag}{name={Incognita},description={valore che deve 
essere determinato}}
\newglossaryentry{infinitog}{name={Infinito},description={quantità maggiore di 
ogni altro valore assegnabile}}
\newglossaryentry{insdisg}{name={Insieme discreto},description={un insieme 
numerico è discreto se fra un numero e il suo successore non vi è nessun altro 
elemento dell'insieme}}
\newglossaryentry{insiemeg}{name={Insieme},description={concetto primitivo, 
raccolta gruppo di elelenti che hanno la stessa proprietà} }
\newglossaryentry{insiemevuotog}{name={Insieme vuoto},description={insieme che 
non ha nessun elemento},see={insiemeg}}
\newglossaryentry{interog}{name={Intero},description={numero che può essere 
espresso come somma o differenza di due numeri naturali},see={numerorelativog}}
\newglossaryentry{intersezioneg}{name={Intersezione},description={punti in 
comune fra più figure},see={figurag}}
\newglossaryentry{intersezioneindisiemig}{name={Intersezione di 
insiemi},description={dati due o più insiemi è l'insieme formato dagli elementi 
in comune fra gli insiemi},see={insiemeg}}
\newglossaryentry{intervallog}{name={Intervallo},description={Insieme ordinato 
di punti compresi tra un punto $a$ che precede tutti gli elementi e un elemento 
$b$ che segue tutti gli elementi. Un intervallo è chiuso se comprende gli 
estremi. Un intorno è aperto se non comprende gli estremi. Un intervallo è 
chiuso a destra ma non a sinistra se comprende l'estremo destro ma non il 
sinistro. Un intervallo è chiuso a sinistra ma non a destra se comprende 
l'estremo sinistro ma non il destro.}}
\newglossaryentry{intervallononlimg}{name={Intervallo non 
limitato},description={un intorno illimitato a destra è l'insieme dei valori 
che seguono un elemento fissato $a$. Un intorno illimitato a sinistra è 
l'insieme dei valori che precedono un valore prefissato $b$}}
\newglossaryentry{intornodespuntog}{name={Intorno destro di un 
punto},description={Un intorno destro di un punto è un intervallo aperto del 
tipo $I^{+}(x_0)=\left(x_0,x_0+\delta\right)$ $\delta>0$},see=[vedi 
anche]{intervallog,intornopuntog}}
\newglossaryentry{intornomenoinfg}{name={Intorno di meno 
infinto},description={Un intorno di un meno infinito è intervallo aperto del 
tipo $I(-\infty)=\left(-\infty,M\right)$ 
$M\in\R$},see={intervallog,intornopuntog}}
\newglossaryentry{intornopiuinfg}{name={Intorno di più infinto},description={Un 
intorno di un più infinito è intervallo aperto del tipo 
$I(+\infty)=\left(M,+\infty\right)$ $M\in\R$},see={intervallog,intornopuntog}}
\newglossaryentry{intornopuntog}{name={Intorno circolare di un 
punto},description={Un intorno circolare di un punto è un intervallo aperto del 
tipo $I(x_0)=\left(x_0-\delta,x_0+\delta\right)$ 
$\quad\delta>0$},see={intervallog}}
\newglossaryentry{intornosinpuntog}{name={Intorno sinistro di un 
punto},description={Un intorno sinistro di un punto è un intervallo aperto del 
tipo $I^{-}(x_0)=\left(x_0-\delta,x_0\right)$ $\delta>0$}, 
see={intervallog,intornopuntog}}
\newglossaryentry{ipotenusag}{name={Ipotenusa},description={in un triangolo 
rettangolo l'ipotenusa è il lato opposto all'angolo retto}}
\newglossaryentry{irg}{type=symbols,name={I},description={Il numero uno nel 
sistema di numerazione romano}}
\newglossaryentry{jg}{type=symbols,name={j},description={unità immaginaria}}
\newglossaryentry{kBg}{name={kilobyte},description={multiplo del byte indica 
$1024$ byte}}
\newglossaryentry{kg}{type=symbols,name={k},description={kilo}}
\newglossaryentry{kilobinaryg}{name={kilobinary},description={pari a $2^{10}$}}
\newglossaryentry{kilog}{name={kilo},description={prefisso moltiplicativo che 
indica $10^3$}}
\newglossaryentry{latoadiacenteg}{name={Lato adiacente},description={In un 
triangolo, un lato è adiacente ad un angolo se il lato è lato dell'angolo}}
\newglossaryentry{latoppostog}{name={Lato opposto},description={In un 
triangolo, un lato è opposto ad un angolo se non è lato dell'angolo}}
\newglossaryentry{lg}{type=symbols,name={l},description={litro}}
\newglossaryentry{limitatoinfg}{name={Limitato inferiormente},description={un 
insieme $A$ è limitato inferiormente se esiste un numero che sia minore o 
uguale ad ogni elemento di $A$},see={estremoinferioreg}}
\newglossaryentry{limitatosupg}{name={Limitato superiormente},description={un 
insieme $A$ è limitato superiormente se esiste un numero che sia maggiore o 
uguale ad ogni elemento di $A$},see={estremosuperioreg}}
\newglossaryentry{lineareg}{name={Lineare},description={di primo grado}}
\newglossaryentry{logaritmog}{name={Logaritmo},description={di un numero reale 
posistivo rispetto ad una base positiva diversa da uno, è l'esponente che 
bisogna dare alla base per ottenere un numero dato}}
\newglossaryentry{lrg}{type=symbols,name={L},description={Il numero cinquanta 
nel sistema di numerazione romano}}
\newglossaryentry{luogogeometricog}{name={Luogo 
geometrico},description={insieme di punti che godono della stessa proprietà}}
\newglossaryentry{maggioranteg}{name={Maggiorante},description={valore maggiore 
o uguale ad ogni elemento di un insieme dato}}
\newglossaryentry{maggioreg}{name={Maggiore},description={relazione di ordine 
tra numeri $a>b$ si legge $a$ maggiore di $b$ se $a-b$ è positivo}}
\newglossaryentry{maggioreougualeg}{name={Maggiore o 
uguale},description={relazione di ordine tra numeri $a\geq b$ si legge $a$ 
maggiore o uguale di $b$ se $a-b$ è positivo o nullo}}
\newglossaryentry{maggioreougualesg}{type=symbols,name={\ensuremath{\geq}},description={maggiore
 o uguale}}
\newglossaryentry{maggioresg}{type=symbols,name={\ensuremath{>}},description={maggiore}}
\newglossaryentry{massimocomundivisoreg}{name={Massimo comune 
divisore},description={intero che divide senza resto due interi dati}}
\newglossaryentry{massimog}{name={Massimo},description={dato un insieme, se 
l'estremo superiore appartiene all'insieme viene chiamato 
massimo},see={estremosuperioreg}}
\newglossaryentry{medianag}{name={Mediana},description={in un triangolo, 
segmento che congiunge un vertice con il punto medio del lato 
opposto},see={puntomediog}}
\newglossaryentry{mediaritmeticag}{name={Media aritmetica},description={in un 
intervallo di valori valore ottenuto dalla loro somma diviso per il numero 
degli elementi}}
\newglossaryentry{mediog}{name={Medio},description={secondo e terzo termine di 
una proporzione}}
\newglossaryentry{medioproporzionaleg}{name={Medio 
proporzionale},description={un numero $b$ è medio proporzionale se esiste una 
proporzione $a\div b=b\div c$}}
\newglossaryentry{megabinaryg}{name={megabinary},description={pari a $2^{20}$}}
\newglossaryentry{megagrammog}{name={Megagrammo},description={unità di misura 
di massa pari a 1000 chilogrammi}}
\newglossaryentry{megag}{name={mega},description={prefisso moltiplicativo che 
indica $10^{6}$}}
\newglossaryentry{metrog}{name={Metro},description={unità di misura della 
lunghezza che fa parte del SI. Un metro è la distanza percorsa dalla luce 
\num{1/299792458} secondi},see={secondog,SI}}
\newglossaryentry{mg}{type=symbols,name={m},description={metro}}
\newglossaryentry{millig}{name={milli},description={prefisso moltiplicativo che 
indica $10^{-3}$}}
\newglossaryentry{minimocomunemultiplog}{name={Minimo comune 
multiplo},description={il più il piccolo numero che è divisibile per tutti i 
numeri di un insieme dati}}
\newglossaryentry{minimog}{name={Minimo},description={dato un insieme, se 
l'estremo inferiore appartiene all'insieme viene chiamato 
minimo},see={estremoinferioreg}}
\newglossaryentry{minoranteg}{name={Minorante},description={valore minore o 
uguale ad ogni elemento di un insieme dato}}
\newglossaryentry{minoreg}{name={Minore},description={relazione di ordine tra 
numeri $a<b$ si legge $a$ minore di $b$ se $b-a$ è positivo}}
\newglossaryentry{minoreougualeg}{name={Minore o uguale},description={relazione 
di ordine tra numeri $a\leq b$ si legge $a$ minore o uguale di $b$ se $b-a$ è 
positivo o nullo}}
\newglossaryentry{minoreougualesg}{type=symbols,name={\ensuremath{\leq}},description={minore
 o uguale}}
\newglossaryentry{minoresg}{type=symbols,name={\ensuremath{<}},description={minore}}
\newglossaryentry{minuendog}{name={Minuendo},description={primo termine 
sottrazione}}
\newglossaryentry{minutog}{name={minuto},description={unità di misura degli 
intervalli di tempo, si compone di sessanta secondi, un minuto è la 
sessantesima parte di un'ora, non accettato dal SI},see={orag,secondog,SI}}
\newglossaryentry{minutosessagesimaleg}{name={Minuto 
sessagesimale},description={è la sessantesima parte di un grado 
sessagesimale, non accettato dal SI},see={gradog,SI}}
\newglossaryentry{modag}{name={Moda},description={in un intervallo di valori 
quello con frequenza maggiore}}
\newglossaryentry{moltiplicazioneg}{name={Moltiplicazione},description={operazione
 binaria}}
\newglossaryentry{monomioformanormaleg}{name={Monomio forma 
normale},description={un monomio è scritto in forma normale se compare un unico 
termine numerico e ogni termine letterale compare una sola volta con un certo 
esponente non negativo}}
\newglossaryentry{monomiogradog}{name={Monomio grado},description={somma degli 
esponenti della parte letterale}}
\newglossaryentry{monomiogradozerog}{name={Monomio grado 
zero},description={monomio formato solo dalla parte numerica}}
\newglossaryentry{monomiog}{name={Monomio},description={monomio espressione 
algebrica in cui non compare ne addizione o sottrazione ma solo il prodotto di 
parti numeriche e letterali che non hanno esponenti negativi}}
\newglossaryentry{monomionullog}{name={Monomio nullo},description={monomio con 
la parte numerica nulla}}
\newglossaryentry{monomiooppostog}{name={Monomio opposto},description={Due 
monomi sono opposti se hanno lo stessa parte letterale ma parte numerica 
opposta},see={monomiosimileg}}
\newglossaryentry{monomioprodottog}{name={Monomio prodotto},description={il 
prodotto di due o più monomi è un monomio che ha per parte letterale il 
prodotto delle parti letteralie e per parte numerica il prodotto delle parti 
numeriche}}
\newglossaryentry{monomiosimileg}{name={Monomio simile},description={due o più 
monomi sono simili se hanno la stessa parte letterale}}
\newglossaryentry{monomiosimilesommag}{name={Monomio somma 
simili},description={la somma algebrica di due monomi simili è un monomio che 
ha per parte numerica la somma algebrica delle parti numeriche e per parte 
letterale la stessa parte letterale},see={monomiosimileg}}
\newglossaryentry{monomiougualeg}{name={Monomio uguale},description={due monomi 
che hanno la stessa parte numerica e la stessa parte letterale sono 
uguali},see={monomiosimileg}}
\newglossaryentry{mrg}{type=symbols,name={M},description={Il numero mille nel 
sistema di numerazione romano}}
\newglossaryentry{msg}{type=symbols,name={m},description={milli}}
\newglossaryentry{multiplog}{name={Multiplo},description={un numero $a$ è 
multiplo di un altro numero $b$ se esiste un numero $c$ tale che $a=b\cdot c$}}
\newglossaryentry{nanogg}{name={nano},description={prefisso moltiplicativo che 
indica $10^{-9}$}}
\newglossaryentry{negativog}{name={Negativo},description={quantità o insieme di 
valori minori di zero}}
\newglossaryentry{ng}{type=symbols,name={n},description={nano}}
\newglossaryentry{nnug}{type=symbols,name={N},description={insieme dei numeri 
naturali}}
\newglossaryentry{norg}{name={nor},description={operatore logico corrisponde 
alla negazione di un or. Vale uno solo se in entrata sono entrambi a zero 
altrimenti è zero},see={org}}
\newglossaryentry{normaleg}{name={Normale},description={perpendicolare ad una 
linea o a un piano}}
\newglossaryentry{notg}{name={not},description={operatore logico. Vale zero se 
in ingresso vale uno e viceversa}}
\newglossaryentry{numcompg}{name={Numero complesso},description={numero nella 
forma ${z=a+\uimm b}$ dove $a$ e $b$ sono numeri reali e $\uimm$ è l'unità 
immaginaria},see={unitaimmaginariag}}
\newglossaryentry{numeratoreg}{name={Numeratore},description={termine che viene 
diviso in una frazione}}
\newglossaryentry{numerireciprocig}{name={Numeri reciproci},description={due 
numeri sono reciproci se il loro prodotto è uno}}
\newglossaryentry{numerodecimaleg}{name={Numero decimale},description={numero 
formato da due parti separate dalla virgola chiamate parte intera e parte 
decimale}}
\newglossaryentry{numerodecimalfinitog}{name={Numero decimale 
finito},description={numero decimale con la parte decimale composta da un 
numero finito di cifre},see={numerodecimaleg}}
\newglossaryentry{numerodecimalinfinitog}{name={Numero decimale 
infinito},description={numero decimale con la parte decimale composta da un 
numero infinito di cifre},see={numerodecimaleg}}
\newglossaryentry{numerodecimalinfinitoperg}{name={Numero decimale 
periodico},description={numero decimale con la parte decimale composta da un 
numero finito di cifre, dette periodo, che si ripetono 
all'infinito},see={numerodecimaleg}}
\newglossaryentry{numerodecimalinfinitopermistg}{name={Numero decimale 
periodico misto},description={numero decimale con la parte decimale divisa in 
una parte finita detta antiperiodo e da un numero finito di cifre, dette 
periodo, che si ripetono all'infinito},see={numerodecimaleg}}
\newglossaryentry{numerodivisibileg}{name={Numero 
divisibile},description={numero che può essere diviso esattamente da un altro 
numero}}
\newglossaryentry{numeroirrazionaleg}{name={Numero 
irrazionale},description={numero non esprimibile come rapporto di due numeri 
interi}}
\newglossaryentry{numeroperiodicog}{name={Numero 
periodico},description={},see={numerodecimalinfinitoperg}}
\newglossaryentry{numeroppostog}{name={Numero opposto},description={Due numeri 
sono opposti se hanno lo stesso valore assoluto ma segno diverso, ovvero se la 
loro somma vale zero}}
\newglossaryentry{numeroprimog}{name={Numero primo},description={numero 
divisibile solo per se stesso e per l'unità}}
\newglossaryentry{numerorazionaleg}{name={Numero razionale},description={numero 
esprimibile come rapporto di due numeri interi}}
\newglossaryentry{numerorelativog}{name={Numero relativo},description={numero 
che può essere positivo, negativo o zero}}
\newglossaryentry{numng}{name={Numeri naturali},description={insieme numerico 
$\Ni=\Set{0,1,2,3,\dots,}$}}
\newglossaryentry{numnprimifralorog}{name={Numeri primi fra 
loro},description={due numeri sono primi fra loro se l'unico numero che li 
divide entrambi è uno}}
\newglossaryentry{obliquog}{name={Obliquo},description={ne parallelo e ne 
perpendicolare}}
\newglossaryentry{oppostog}{name={Opposto},description={ha vari 
significati},see={monomiooppostog,numeroppostog,latoppostog,antiorariog}}
\newglossaryentry{orag}{name={Ora},description={unità di misura degli 
intervalli di tempo che non fa parte del SI, un ora è sessanta minuti o $3600$ 
secondi},see={secondog,SI}}
\newglossaryentry{orariog}{name={Orario},description={movimemento circolare che 
avviene come quello delle lancette dell'orologio}}
\newglossaryentry{org}{name={or},description={operatore logico. Vale zero solo 
se in entrata sono entrambi a zero altrimenti vale uno}}
\newglossaryentry{orsdg}{type=symbols,name={\ensuremath{+}},description={or}}
\newglossaryentry{orsg}{type=symbols,name={\ensuremath{\lor}},description={or}}
\newglossaryentry{ortocentrog}{name={Ortocentro},description={in un triangolo 
punto in comune tra le altezze o le rette a cui appartengono},see={altezzag}}
\newglossaryentry{ortogonaleg}{name={Ortogonale},description={Peperpendicolare,normale},see={normaleg,perpendicolareg}}
\newglossaryentry{parallelismog}{name={Parallelismo},description={relazione che 
può verificarsi fra due rette, due piani, fra una retta e un piano che non 
hanno punti in comune, tra due curve}}
\newglossaryentry{parallelogrammag}{name={Parallelogramma},description={quadrilatero
 con quattro lati paralleli e congruenti},see={congruenzag} }
\newglossaryentry{parig}{name={Pari},description={numero divisibile per due}}
\newglossaryentry{periodofun}{name={Periodo funzione},description=rappresenta 
il più piccolo valore $T$ per cui ${f(x)=f(x+T)}$ per ogni x}
\newglossaryentry{periodonum}{name={Periodo numero},description={in un numero 
decimale indica una sequenza di cifre che si ripete ciclicamente nella sua 
rappresentazione},see={numerodecimaleg}}
\newglossaryentry{perpendicolareg}{name={Perpendicolare},description={che forma 
un angolo retto}}
\newglossaryentry{pesospecificog}{name={Peso specifico},description={rapporto 
tra il peso di un corpo e il suo volume}}
\newglossaryentry{pgg}{type=symbols,name={\ensuremath{\pi}},description={rapporto
 tra la circonferenza e il suo diametro}}
\newglossaryentry{pgrecog}{type=symbols,name={\ensuremath{\pi}},description={numero
 decimale infinito non periodico}}
\newglossaryentry{pg}{type=symbols,name={p},description={pico}}
\newglossaryentry{pianoArgandGaussg}{name={Piano di Argand 
Gauss},description={rappresentazione geometrica dell'insieme dei numeri 
complessi $\Co$}}
\newglossaryentry{picog}{name={pico},description={prefisso moltiplicativo che 
indica $10^{-12}$}}
\newglossaryentry{pigrecog}{name={Pi greco},description={numero irrazionale 
uguale al rapporto tra la circonferenza e il suo diametro}}
\newglossaryentry{polinomiocoefficienteg}{name={Polinomio 
coefficiente},description={parti numeriche dei monomi che lo compongono}}
\newglossaryentry{polinomioformanormaleg}{name={Polinomio forma 
normale},description={un polinomio è in forma normale quando è somma algebrica 
di monomi non simili},see={monomiosimileg}}
\newglossaryentry{polinomiogradog}{name={Polinomio grado},description={il grado 
di un polinomio non nullo e in forma normale è il massimo dei gradi dei monomi 
che lo compongono},see={polinomionullog,monomiogradog}}
\newglossaryentry{polinomiog}{name={Polinomio},description={somma algebrica di 
due o più monomi}}
\newglossaryentry{polinomionullog}{name={Polinomio nullo},description={un 
polinomio è nullo se tutti i termini sono zero}}
\newglossaryentry{polinomioomogeneog}{name={Polinomio 
omogeneo},description={polinomio in cui tutti i suoi addendi sono dello stesso 
grado},see={polinomiog}}
\newglossaryentry{polinomiougualeg}{name={Polinomio uguale},description={due 
polinomi sono uguali se ridotti in forma normale, hanno uguali tutti i 
termini},see={polinomioformanormaleg}}
\newglossaryentry{polinomiterminenotog}{name={Polinomio termine 
noto},description={in un polinomio non nullo e in forma normale è il termine se 
presente, di grado zero. Se non è presente il termine noto è uguale a 
zero},see={monomiogradog,polinomioformanormaleg}}
\newglossaryentry{potenzag}{name={Potenza},description={risultato 
dell'elevamento a potenza},see={elevamentoapotenzag}}
\newglossaryentry{primog}{name={Primo},description={numero divisibile solo per 
se stesso e per l'unità}}
\newglossaryentry{primomembroequazioneg}{name={Primo membro},description={in 
una equazione è il termine che si trova a sinistra del segno di uguaglianza}}
\newglossaryentry{primoprincipioequig}{name={Primo principio 
equivalenza},description={sommando o sottraendo la stessa quantità al primo e 
al secondo membro di una equazione si ottiene una equazione equivalente a 
quella data}}
\newglossaryentry{prodottog}{name={Prodotto},description={risultato 
moltiplicazione}}
\newglossaryentry{puntiallineatig}{name={Punti allineati},description={punti 
che appartengono alla stessa retta}}
\newglossaryentry{puntomediog}{name={Punto medio},description={punto 
equidistante dagli estremi di un segmento che divide in due parti uguali}}
\newglossaryentry{qnug}{type=symbols,name={Q},description={insieme dei numeri 
razionali}}
\newglossaryentry{quadranteg}{name={Quadrante},description={una delle quattro 
parti in cui un sistema di riferimento cartesiano ortogonale divide il piano. I 
quadranti sono numerati a partire dal semiasse positivo delle ascisse e delle 
ordinate e sono numerati in senso antiorario}}
\newglossaryentry{quadratog}{name={Quadrato},description={figura geometrica che 
ha quattro lati uguali e quattro angoli retti}}
\newglossaryentry{quozienteg}{name={Quoziente},description={risultato della 
divisione}}
\newglossaryentry{radianteg}{name={radiante},description={un angolo al centro 
ha l'ampiezza di un radiante quando sottende sulla circonferenza un arco uguale 
al raggio della circonferenza}}
\newglossaryentry{raggiog}{name={Raggio},description={segmento che unisce il 
centro di un cerchio con un punto sulla circonferenza, il raggio è la metà del 
diametro},see={diametrog}}
\newglossaryentry{rapportoincrementaleg}{name={Rapporto 
incrementale},description={data una funzione $\funzione{f}{\R}{\R}$, a valori 
reali, il rapporto incrementale è il quoziente  $\Delta f/\Delta 
x=\left(f(x+h)-f(x)\right)/h$ }}
\newglossaryentry{regolacang}{name={Regola cancellazione},description={se lo 
stesso termine è al primo e al secondo membro di un'equazione, allora può 
essere cancellato}}
\newglossaryentry{regolatraspg}{name={Regola trasporto},description={spostando 
un termine dal primo al secondo membro di un'equazione e viceversa bisogna 
cambiargli di segno}}
\newglossaryentry{rettangolog}{name={Rettangolo},description={quadrilatero con 
i lati uguali due a due e che formano quattro angoli retti}}
\newglossaryentry{rettasecanteg}{name={Secante},description={retta che 
interseca una retta in più punti}}
\newglossaryentry{rettatangenteg}{name={Retta tangente},description={retta che 
interseca una retta in un punto}}
\newglossaryentry{rettecoincidentig}{name={Rette 
coincidenti},description={rette che hanno due punti in comune}}
\newglossaryentry{retteincidentig}{name={Rette incidenti},description={rette 
che hanno un punto in comune}}
\newglossaryentry{retteparalleleg}{name={Rette parallele},description={rette 
dello stesso piano che non hanno punti in comune oppure sono coincidenti}}
\newglossaryentry{retteperpendicolarig}{name={Rette 
perpendicolari},description={rette incidenti che formano quattro angoli 
retti},see={retteincidentig}}
\newglossaryentry{risequag}{name={Risolvere un'equazione},description={trovare 
le soluzioni dell'equazione}}
\newglossaryentry{rnug}{type=symbols,name={R},description={insieme dei numeri 
reali}}
\newglossaryentry{rombog}{name={Rombo},description={quadrilatero con i lati 
uguali e che hanno gli angoli opposti uguali}}
\newglossaryentry{scompfatprimig}{name={Scomposizione in fattori 
primi},description={scrivere un numero come prodotto di numeri primi}}
\newglossaryentry{secondog}{name={secondo},description={unità di misura degli 
intervalli di tempo che fa parte del SI. Un secondo è la durata di 
\num{9192631770} oscillazioni della radiazione emessa dall'atomo di 
cesio~133},see={SI}}
\newglossaryentry{secondomembroequazioneg}{name={Secondo 
membro},description={in una equazione è il termine che si trova a destra del 
segno di uguaglianza}}
\newglossaryentry{secondoprincipioequig}{name={Secondo principio 
equivalenza},description={moltiplicando o dividendo per la stessa quantità 
diversa da zero il primo e il secondo membro di una equazione si ottiene una 
equazione equivalente a quella data}}
\newglossaryentry{secondosessagesimaleg}{name={Secondo 
sessagesimale},description={è la sessantesima parte di un minuto o la 3600 
parte di un grado sessagesimale, non accettato dal SI},see={gradog,SI}}
\newglossaryentry{segmentoadiacenteg}{name={Segmento 
adiacente},description={segmenti che hanno un estremo in comune ed appartengono 
alla stessa retta}}
\newglossaryentry{segmentoconsecutivog}{name={Segmento 
consecutivo},description={segmenti che hanno un estremo in 
comune},see={segmentoadiacenteg}}
\newglossaryentry{segmentog}{name={Segmento},description={parte di retta 
compresa fra due suoi punti}}
\newglossaryentry{semicirconferenzag}{name={Semicirconferenza},description={ciascuna
 delle due parti in cui un diametro divide una circonferenza},see={diametrog}}
\newglossaryentry{semipianog}{name={Semipiamo},description={ciascuna delle due 
parti in cui una retta divide un piano}}
\newglossaryentry{semirettag}{name={Semiretta},description={ciascuna delle due 
parti in cui un punto divide una retta}}
\newglossaryentry{senog}{name={Seno},description={in un triangolo rettangolo il 
seno di un anglo acuto è uguale al rapporto tra il cateto opposto all'angolo e 
l'ipotenusa. In una circonfereza goniometrica, il seno di un angolo è 
l'ordinata del punto individuato dall'angolo sulla 
circonferenza},see={oppostog,circgoniog,ipotenusag,catetog}}
\newglossaryentry{separazionevarg}{name={Separare le 
variabili},description={tecnica che consiste nel portare al primo membro di 
un'equazione le incognite e al secondo membro i rimanenti termini}}
\newglossaryentry{simmetriaasialeg}{name={Simmetria assiale},description={una 
simmetria assiale di asse $r$ è una trasformazione che ad ogni punto $P$ fa 
corrispondere il punto $P'$ in modo che: il segmento $PP'$ sia perpendicolare 
alla retta $r$ e che il punto medio del segmento $PP'$ appartenga alla retta 
$r$}}
\newglossaryentry{simmetriacentraleg}{name={Simmetria 
centrale},description={una simmetria di centro $C$ è una trasformazione che ad 
ogni punto $P$ fa corrispondere un punto $P'$ in modo che il centro $C$ sia il 
punto medio del segmento $PP'$}}
\newglossaryentry{sistemaequag}{name={Sistema},description={risolvere 
contemporaneamente due o più equazioni}}
\newglossaryentry{sistemaimpossibileg}{name={Sistema 
impossibile},description={sistema di equazioni che non ha soluzioni in comune a 
tutte le equazioni che lo compongono}}
\newglossaryentry{sistemaindeterminatog}{name={Sistema 
indeterminato},description={sistema di equazioni che ha infinte soluzioni in 
comune a tutte le equazioni che lo compongono}}
\newglossaryentry{sistemascisseg}{name={Sistema di 
ascisse},description={abbiamo su una retta un sistema di ascisse se sulla retta 
definiamo, un verso, un'origine ed un'unità di misura. Ad ogni punto della 
retta viene associato un valore numerico e viceversa}}
\newglossaryentry{soluzionequazioneg}{name={Soluzione},description={valore che 
sostituito all'incognita, rende vera l'equazione}}
\newglossaryentry{soluzionesistemag}{name={Soluzione sistema},description={una 
soluzione per un sistema è un insieme ordinato di valori che sono soluzione per 
ogni equazione del sistema}}
\newglossaryentry{sommag}{name={Somma},description={risultato dell'addizione}}
\newglossaryentry{sottoindisiemig}{name={Sottoinseme},description={dati due 
insiemi, diremo che un insieme è un sotto insieme dell'altro se tutti i suoi 
elementi appartenogono all'altro},see={insiemeg}}
\newglossaryentry{sottraendog}{name={Sottraendo},description={secondo termine 
sottrazione}}
\newglossaryentry{sottrazioneg}{name={Sottrazione},description={operazione 
binaria}}
\newglossaryentry{tangentegoniog}{name={Tangente goniometrica},description={in 
un triangolo rettangolo la tangente di una angolo acuto è il rapporto tra il 
cateto opposto e il cateto adiacente allo stesso angolo}}
\newglossaryentry{teopitagorag}{name={Teorema di Pitagora},description={in un 
triangolo rettangolo il quadrato costruito sull'ipotenusa è equivalente alla 
somma dei quadrati costruiti su i cateti, 
$a^2=b^2+c^2$},see={catetog,ipotenusag}}
\newglossaryentry{terabinaryg}{name={terabinary},description={pari a $2^{40}$}}
\newglossaryentry{terabyteg}{name={terabyte},description={pari a $10^{12}$byte}}
\newglossaryentry{terminenotog}{name={Termine noto},description={in un 
polinomio non nullo e in forma normale è il termine se presente, di grado zero. 
Se non è presente il termine noto è uguale a zero},see={polinomiterminenotog}}
\newglossaryentry{ternaPitag}{name={Terna pitagorica},description={tre numeri 
legati dalla relazione di Pitagora}}
\newglossaryentry{tg}{type=symbols,name={t},description={tonnellata}}
\newglossaryentry{tonnellatag}{name={Tonnellata},description={unità di misura 
di massa pari a 1000 chilogrammi. Non fa parte del SI.},see={SI}}
\newglossaryentry{trapeziog}{name={Trapezio},description={quadrilatero con due 
lati paralleli dette basi e i rimanenti lati obliqui}}
\newglossaryentry{trapezioisog}{name={Trapezio isoscele},description={trapezio 
con i lati obliqui uguali},see={trapeziog}}
\newglossaryentry{trapezioretg}{name={Trapezio 
rettangolo},description={trapezio con un lato obliquo perpendicolare alle 
basi},see={trapeziog}}
\newglossaryentry{triangoloeqig}{name={Triangolo 
equilatero},description={triangolo con tre lati uguali}}
\newglossaryentry{triangolog}{name={Triangolo},description={poligono ottenuto 
da tre punti non allineati, ha tre lati e tre vertici}}
\newglossaryentry{triangoloisog}{name={Triangolo 
isoscele},description={triangolo con due lati uguali}}
\newglossaryentry{triangolorettangolog}{name={Triangolo 
rettangolo},description={triangolo in cui due lati formano un angolo 
retto},see={angolorettog}}
\newglossaryentry{triangoloscag}{name={Triangolo 
scaleno},description={triangolo con tre lati disuguali}}
\newglossaryentry{unitaimmaginariag}{name={Unità 
immaginaria},description={particolare numero complesso definito come 
$\uimm^2=-1$}}
\newglossaryentry{variabileg}{name={Variabile},description={una variabile è un 
carattere che rappresenta una quantità numerica non nota}}
\newglossaryentry{versoreg}{name={Versore},description={vettore di modulo 
unitario usato per indicare una particolare direzione e verso}}
\newglossaryentry{vettoreg}{name={Vettore},description={un vettore in geometria 
è un segmento dotato di una direzione, un verso, un modulo}}
\newglossaryentry{vrg}{type=symbols,name={V},description={Il numero cinque nel 
sistema di numerazione romano}}
\newglossaryentry{xrg}{type=symbols,name={X},description={Il numero dieci nel 
sistema di numerazione romano}}
\newglossaryentry{znug}{type=symbols,name={Z},description={insieme dei numeri 
interi}}